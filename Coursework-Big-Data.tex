\documentclass{article}

\usepackage{Sweave}
\begin{document}
\input{Coursework-Big-Data-concordance}

\Title{1504107 - Coursework}

\subTitle{1. Data Exploration}

\subTitle{1.1 Dataset Choice}

The dataset "Breast Cancer Wisconsin (Original)" was choosen and was obtained from UC Irvine Machine Learning Repository website. 

I choose the "Breast Cancer Wisconsin (Original)" dataset as I believe I would be able to find solid relationships between the classes of tumors and the nine contrubuting factors within the dataset. 


\subTitle{1.2 Problem Statement & Data Exploration}

The choosen dataset contains records of patients with tumors, the tumors are classafied as either 'Benign' or 'Malignant'. Each tumor sample, uniquely identified using a 'Sample code number', is given various values from 1-10 on varing factors including 'Clump Thickness' and 'Mitoses'. The aim is to predict wether a tumor will be 'benign' or 'melignant' depending on these varying factors. 

The dataset will be used to build a model to predict if a tumor is either melignant or benign. Using all nine of the other factor, the model created should be able to predict, with a fair degree of accuracy, wether or not a sample tumor is melignant or benign.

\begin{Schunk}
\begin{Sinput}
> install.packages("randomForest")