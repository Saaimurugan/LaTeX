%%%%%%%%%%%%%%%%%%%%%%%%%%%%%%%%%%%%%%%%%%%%%%%%%%%%%%%%%%%%%%%%%%%%%
% Documentclass
% -------------------
% Defines the documentclass
%%%%%%%%%%%%%%%%%%%%%%%%%%%%%%%%%%%%%%%%%%%%%%%%%%%%%%%%%%%%%%%%%%%%%
\documentclass[a4paper,11pt,titlepage,fleqn]{report}



\usepackage[usenames, dvipsnames]{color}
\usepackage[table]{xcolor}
%%%%%%%%%%%%%%%%%%%%%%%%%%%%%%%%%%%%%%%%%%%%%%%%%%%%%%%%%%%%%%%%%%%%%
% Needed packages
% -------------------
% Inclusion and configuration of all packages that are needed in the 
% document
%%%%%%%%%%%%%%%%%%%%%%%%%%%%%%%%%%%%%%%%%%%%%%%%%%%%%%%%%%%%%%%%%%%%%
% -------------------------------------
% Encoding
% -------------------------------------
\usepackage[utf8]{inputenc}

% -------------------------------------
% German translations and other audjustments
% -------------------------------------
\usepackage[english,ngerman]{babel}

% -------------------------------------
% Quotation signs
% http://www.tug.org/texlive/Contents/live/texmf-dist/doc/latex/csquotes/csquotes.pdf
% -------------------------------------
% German style (Example: "Quote")
% \usepackage[babel,german=quotes]{csquotes}
% Swiss style (Example: >>Quote<<)
\usepackage[babel,german=guillemets]{csquotes}
% French style (Example: <<Quote>>)
%\usepackage[babel,german=swiss]{csquotes}

% -------------------------------------
% Use T1 fonts
% -------------------------------------
\usepackage[T1]{fontenc}

% -------------------------------------
% Font configuration (See http://www.tug.dk/FontCatalogue)
% Please make sure that you just use T1 fonts here. Recommended fonts
% are the following:
%   - Computer Modern (Serif)
%   - Adobe Utopia Math Design (Serif)
%   - Charter BT (Serif)
% -------------------------------------
% \usepackage{lmodern}
%\usepackage[adobe-utopia]{mathdesign}
\usepackage[bitstream-charter]{mathdesign}

% -------------------------------------
% Typewriter font configuration (See http://www.tug.dk/FontCatalogue)
% Please make sure that you just use T1 fonts here.
% -------------------------------------
\usepackage[scaled]{beramono}
%\usepackage{inconsolata}
%\usepackage{DejaVuSansMono}

% -------------------------------------
% Used to adjust the margins of the page
% http://ctan.mackichan.com/macros/latex/contrib/anysize/anysize.pdf
% -------------------------------------
\usepackage{anysize}
\usepackage[strict]{changepage}

% -------------------------------------
% Used to adjust the title of chapters and the table of contents
% ftp://tug.ctan.org/pub/tex-archive/macros/latex/contrib/titlesec/titlesec.pdf
% -------------------------------------
\usepackage{titlesec}
\usepackage{titletoc}

% -------------------------------------
% Used to adjust the header and footer of the page
% http://ftp.gwdg.de/pub/ctan/macros/latex/contrib/fancyhdr/fancyhdr.pdf
% -------------------------------------
\usepackage{fancyhdr}

% -------------------------------------
% URLs
% -------------------------------------
\usepackage[hyphens]{url}

% -------------------------------------
% PDF improvments like internal hyperlinks and bookmarks
% -------------------------------------
\usepackage[
    bookmarks=true,
    bookmarksopen=true,
    bookmarksnumbered=true,
    pdfstartpage=1,
    pdftitle={},
    pdfauthor={\student},
    breaklinks=true,
% Uncomment these lines to avoid that the links are marked with boxes
    colorlinks=true,
    linkcolor=black,
    anchorcolor=black,
    citecolor=black,
    filecolor=black,
    menucolor=black,
    urlcolor=black,
]
{hyperref}

% -------------------------------------
% Packages for graphics
% -------------------------------------
\usepackage{graphicx}
\usepackage{graphics} 
\usepackage{color}

% -------------------------------------
% Packages to create high-quality tabulars
% -------------------------------------
\usepackage{longtable}
\usepackage{booktabs}

% -------------------------------------
% Package to create code listings
% ftp://tug.ctan.org/pub/tex-archive/macros/latex/contrib/titlesec/titlesec.pdf 
% -------------------------------------
\usepackage{listings}

% -------------------------------------
% Inclusion of pdf documents
% -------------------------------------
\usepackage{pdfpages}
\usepackage{pdflscape} 

% -------------------------------------
% Multiple columns (used for the abstract)
% -------------------------------------
\usepackage{multicol}

% -------------------------------------
% Citations
% -------------------------------------
\usepackage[numbers]{natbib}

% -------------------------------------
% Glossary
% -------------------------------------
\usepackage[nonumberlist, acronym]{glossaries}
\renewcommand*{\glspostdescription}{}

% -------------------------------------
% Acronyms
% -------------------------------------
%\usepackage[printonlyused]{acronym}
\usepackage{nomencl}

% -------------------------------------
% Mathematic equations
% -------------------------------------
\usepackage{amsmath}
%\usepackage{txfonts}

% -------------------------------------
% Footnotes
% -------------------------------------
\usepackage[singlespacing]{setspace}
\usepackage{scrextend}
\usepackage{chngcntr}

%color



%%%%%%%%%%%%%%%%%%%%%%%%%%%%%%%%%%%%%%%%%%%%%%%%%%%%%%%%%%%%%%%%%%%%%
% Document adjustments
% -------------------
% Audjustments that have influence on the document generation
%%%%%%%%%%%%%%%%%%%%%%%%%%%%%%%%%%%%%%%%%%%%%%%%%%%%%%%%%%%%%%%%%%%%%
% -------------------------------------
% Adjusts the baselinestretching and the spacing between tabular lines
% -------------------------------------
\renewcommand{\baselinestretch}{1.2}\normalsize
\renewcommand{\arraystretch}{1.25}

% -------------------------------------
% Avoids that 'Chapter X' is printed
% See package titlesec for more information
% -------------------------------------
\titleformat{\chapter}{\bf\LARGE}{\thechapter\quad}{0em}{}
\titlespacing{\chapter}{0pt}{-2em}{1em}

% -------------------------------------
% Shows the current chapter and the pagenumber in the header
% See package fancyhdr for more information
% -------------------------------------
\pagestyle{fancy}
\renewcommand{\chaptermark}[1]{\markboth{\textsc{\thechapter\ #1}}{}}
\fancyhead{}
\fancyfoot{}
\lhead{\nouppercase{\textsc{\leftmark}}}
\rhead{\thepage}
\fancypagestyle{plain}{%
	\fancyhead{}
	\fancyfoot{}
	\lhead{\nouppercase{\textsc{\leftmark}}}
	\rhead{\thepage}
}

% -------------------------------------
% Decrease spacing between chapters in table of contents
% -------------------------------------
\addtocontents{toc}{\vspace{-3ex}}

% -------------------------------------
% Chapter numbering und the table of contents should be done till 
% subsubsections
% -------------------------------------
\setcounter{secnumdepth}{4}
\setcounter{tocdepth}{2}

% -------------------------------------
% Adjusts the margins of the page
% Uncomment the line to get the standard margins
% See package anysize for more information
% -------------------------------------
%\marginsize{3cm}{3cm}{2cm}{2cm}

% -------------------------------------
% Changes the default typewriter font (used for texttt and code 
% listings) to TXTT
% See http://www.tug.dk/FontCatalogue/typewriterfonts.html for more 
% fonts
% -------------------------------------
%\renewcommand*\ttdefault{txtt}

% -------------------------------------
% Define colors that can be used in the document (used for syntax 
% highlighting in code listings)
% -------------------------------------
\definecolor{lightgray}{rgb}{0.95,0.95,0.95}
\definecolor{darkblue}{rgb}{0,0,0.75}
\definecolor{darkred}{rgb}{0.75,0,0}
\definecolor{darkgreen}{rgb}{0,0.75,0}

% -------------------------------------
% Configuration for code listings
% See package listings for more information
% -------------------------------------
\lstset{
    basicstyle=\ttfamily\small\mdseries,
    keywordstyle=\color{darkblue}\bf,%\bfseries,
    identifierstyle=,
    commentstyle=\color[rgb]{.133,.545,.133},%\color{darkgreen},
    stringstyle=\color{darkred},
    numbers=left,
    numberstyle=\tiny\ttfamily,
    % Defines the interval between the line numbers
    stepnumber=1,
    numberfirstline=false,
    firstnumber=1,
    breaklines=true,
    frame=single,
    framerule=0pt,
%    frameshape={nnn}{n4}{n}{nnn},
    backgroundcolor=\color{lightgray},
    showstringspaces=false,
    tabsize=4,
    captionpos=b,
    float=htbp,
    xleftmargin=3ex,
    xrightmargin=3ex,
    belowcaptionskip=2ex,
    escapeinside={@}{@},
    breakatwhitespace=true,
    breakautoindent=true
}

% -------------------------------------
% Use the quotation signs configured in csquotes
% -------------------------------------
\defineshorthand{"`}{\openautoquote}
\defineshorthand{"'}{\closeautoquote}

% -------------------------------------
% Footnote format
% -------------------------------------
\deffootnote[1em]{1em}{1em}{\textsuperscript{\thefootnotemark\ }}
\counterwithout{footnote}{chapter}

% -------------------------------------
% Acronyms
% -------------------------------------
% German title
\renewcommand{\nomname}{Abkürzungsverzeichnis}
% Print dots between acronym and description
\setlength{\nomlabelwidth}{.20\hsize}
\renewcommand{\nomlabel}[1]{#1 \dotfill}
% Decrease line height
\setlength{\nomitemsep}{-\parsep}

% --- Abkürzung bei der ersten Verwendung ------------------
%Zum Abkürzungsverzeichnis hinzufügen
%Neuen Anker für die Abkürzung setzen
%Link auf Abkürzungsverzeichnis, Abkürzung und Bedeutung ausschreiben
\newcommand{\abkDef}[2]{\nomenclature{#1}{#2}\label{Abk.#1}\hyperref[AbkVz]{#1 (#2)}}

% --- Abkürzung bei weiterer Verwendung --------------------
\newcommand{\abk}[1]{\hyperref[Abk.#1]{#1}} 

\definecolor{light-gray}{gray}{0.90}
\newcommand{\clstinline}[1]{\colorbox{light-gray}{\lstinline|#1|}}