\documentclass[12pt,reqno]{amsart} %\pagestyle{empty}
\mathsurround=3pt
\parindent=20pt
\usepackage{amsmath}
\usepackage{amssymb}
\usepackage{amsthm}
\usepackage{hyperref}
\usepackage{cleveref}
\usepackage{tikz}
\usepackage{caption}
\usepackage{setspace}
\usepackage{pst-all}
\doublespacing

%\baselineskip=1.5in
\baselineskip=1.5in
\textheight 650pt
\textwidth 400pt

\oddsidemargin 3pt
\evensidemargin 3pt
\topmargin0pt
\usepackage[colorinlistoftodos,bordercolor=orange,backgroundcolor=orange!20,linecolor=orange,textsize=small,disable]{todonotes}

\newtheorem{thm}{Theorem}
\newtheorem{lem}[thm]{Lemma}
\newtheorem{prop}[thm]{Proposition}
\newtheorem{cor}[thm]{Corollary}
\theoremstyle{definition}
\newtheorem*{definition}{Definition}
\newcommand{\R}{\mathbb{R}}
\newcommand{\Z}{\mathbb{Z}}
\newcommand{\Q}{\mathbb{Q}}
%\newcommand{\L}{\mathbb{L}}
\newcommand{\u}{\mathbf{u}}
\newcommand{\x}{\mathbf{x}}
\newcommand{\y}{\mathbf{y}}
\newcommand{\w}{\mathbf{w}}
\newcommand{\closure}{\operatorname{clo}}
\newcommand{\interior}{\operatorname{int}}
\newcommand{\boundary}{\operatorname{bdry}}



\def \qed   {\hfill \vrule height6pt width 6pt depth 0pt}
\title[]{The directed spanning forest on a perturbed lattice}
\author{}
\address{}
\email{rajendra.bhatia@ashoka.edu.in}

\begin{document}
\maketitle



\begin{abstract}
\end{abstract}

\vspace{0.1in}
\noindent
{\bf Key words:} Markov chain, Random
walk, Directed spanning forest, Brownian web, Perturbed lattice.

\vspace{0.1in}
\noindent
{\bf AMS 2000 Subject Classification:} 60D05.



\begin{figure}[httb]
\begin{center}
\psset{unit=.8}
\begin{pspicture}(0,-0.5)(11,3)
\psline(0,0)(14,0)
\pspolygon[fillcolor=gray,fillstyle=solid](0,0)(9.5,0)(9.5,3)(0,3)
\pspolygon[fillcolor=gray,fillstyle=solid](11,0)(14,0)(14,3)(11,3)
\pscircle[fillcolor=black,fillstyle=solid](3,0){.13}
\pscircle[fillcolor=black,fillstyle=solid](8,0){.13}
\psline[linestyle=dashed](0,0)(0,3)
\psline[linestyle=dashed](6,0)(6,3)
\psline[linestyle=dashed](5,0)(5,3)
\pscircle[fillcolor=black,fillstyle=solid](10.3,1.8){.13}
% \psline{<->}(8.1,3.2)(10.2,3.2)
\rput(9.0,3.44){$I^{(v)}_1 $}
\rput(3.0,3.43){$I^{(0)}_1 = 0$}
% \rput(3.0,-0.4){$(0,0)$}
% \rput(6.3,-0.35){$(\frac{1}{2},0)$}
% \rput(0.0,-0.4){$(-\frac{1}{2},0)$}
% \rput(8.0,-0.4){$(v,0)$}
% \rput(4.65,-0.4){$(v-\frac{1}{2},0)$}
% \rput(11.3,-0.4){$(v+\frac{1}{2},0)$}
% \rput(9.5,-0.4){$(v+\frac{1}{4},0)$}
% \rput(14,-0.4){$(v+1,0)$}
\end{pspicture}
\end{center}

\label{fig:EventAPoisson}
\caption{Lower bound for $\mathbb{P}(A)$}
\end{figure}

\end{document}
